\documentclass[12pt,letterpaper]{article}

\title{Debunking Cantor}
\author{Ghassan Misherghi}

\begin{document}

\maketitle

\begin{abstract}
We show that both of Cantor's arguments that the reals and
$\mathcal{P}{(\infty)}$ have greater cardinality than the natrual numbers to
be false. We show that there is likely no infinity with greater cardinality than
the naturals. We show that there are no real numbers. We then defy Russel and
show that the set of all sets does in fact contain itself. We then muse on the
infinite and God.
\end{abstract}

To debunk Cantor's argument that the powerset of infinity is greater than
itself we only need to show that $\infty = 2^{\infty}$. So let us look at
$\lg\infty$. Suppose $\lg\infty$ was finite. Then $\infty$ would also be finite,
a contradiction. So $\lg\infty$ must be infinite. This means it is just as big
as all the naturals, so we have that $\lg\infty=\infty$. We therefore have that
$\infty=2^{\lg\infty}=2^{\infty}$. This method is suprisingly general, working
with other increasing functions and their inverses.

We also show the same result with a one to one mapping of
$\mathcal{P}{(\infty})$ to $\infty$. Consider the binary representation of the
naturals. We can map any subset of the naturals to a binary number such that
for any element present in the subset, denoted $i$, the $i$th digit is set to
$1$, and $0$ otherwise. This mapping shows that for any subset of the naturals
there is exactly one number it is mapped to and vice versa. Since there is no
maximum natural this mapping will always hold. 

Now let us consider Cantor's argument that the set of reals is a geater infinity
than the set of naturals. Recall that Cantor constructs a real number differing
than the diaganol of an ordering of reals. This supposedly shows that the number
is not in the set of infinite numbers that make up the diagonal, showing that it
is larger than $\infty$. We argue that if the diagonal is infinite then one
more number for which the diagonal is the same is an equivalent diagonal.
Moreover we have that $\frac{1}{\infty}=\frac{2}{2\infty}=\frac{2}{\infty}$.
Also, we can see that
$\frac{1}{\infty}=\frac{2-1}{\infty}=\frac{1-1}{\infty}=\frac{0}{\infty}$.  This
implies real numbers are always the same at infintismal places.

We propose that this implies there is no greater infinity. See that
$\frac{1}{\infty}=0$. Then it must be the case that $\frac{1}{\aleph_1}=0$. We
now have that $\infty^{-1}=\aleph_1^{-1}$. It follows that
$(\infty^{-1})^{-1}=(\aleph_1^{-1})^{-1}$ thus we conclude that
$\infty=\aleph_1$. It is worth nothing that if the order of application is done
wrongly, this results in a divide by 0. This does lead credence to dividing by
zero resulting in $\infty$.

We also propose that there are no real numbers. Since the infinitismal places of
a real number are always $0$, this means significant digits must be at finite
places only. This implies the real number is actually a rational number.

Using set theoretic naturals, we need that $\infty$ contains $\infty-1$.  But
this is $\infty$ so we conclude that $\infty\in\infty$. No typing logic is
needed to show that the set of all sets contains itself. We also conclude that
$\infty>\infty$.

We have discredited Cantor's most famous arguments. As evidence, if there were
always a greater infinity the probability of being created would be $0$, unlike
the probability of $1$ we would expect otherwise if God exists. As a fun aside,
I advance that creation requires math, and the statement, "I must save", is a
mathmatical statement that is true, and therefore necessary for existence.  We
posit that God is really just mathmatical truth, and by living we are witnessing
our very own personalized theorem.

\begin{thebibliography}{9}
\bibitem{cantor1}
G. Cantor (1874) \emph{Über eine Eigenschaft des Inbegriffes aller reellen
algebraischen Zahlen}, Crelles Journal f. Mathematik 77 258 - 262.

\bibitem{cantor2}
G. Cantor (1879) \emph{Über unendliche lineare Punktmannigfaltigkeiten} Nr. 1,
Math.  Ann. 15 1-7.

\bibitem{cantor3}
G. Cantor (1890 - 1891) \emph{Über eine elementare Frage der
Mannigfaltigkeitslehre} Jahresbericht der Deutschen Math. Vereinigung I 75 -
78.

\bibitem{cantor4}
G. Cantor (1882) \emph{Über unendliche lineare Punktmannigfaltigkeiten} Nr. 3,
Math. Ann. 20 113-121. 

\end{thebibliography}

\end{document}
